\Objet{Tâche}
{
	
}
{
\begin{itemize}
	\item id : \texttt{int}
	\item etat : \texttt{enum\{TODO\}} 
	\item priorite : \texttt{int} 
\end{itemize}
}
{
\paragraph{Primitives :}
Toutes les primitives avec un type de retour entier, renvoient -1 en cas d'erreur (\textsl{e.g} plus de mémoire disponible, identifiant invalide, \textsl{etc.}). Sauf indication contraire, elles renvoient 0 en cas de succès.

\begin{itemize}
	\item \texttt{int creer\_tache( void(*) fonction(), int priorite, void* n1, void* n2, ..., void* n10)} : Crée une nouvelle tâche appelant la fonction \texttt{fonction} passée en paramètres, et dont les paramètres d'appels sont les valeurs \texttt{n1} à \texttt{n10}. Si la fonction à appeller prend moins de 10 paramètres, les paramètres inutiles sont ignorés. La priorité de la tâche crée est celle passée en paramètre. En cas de succès, son identifiant est retourné. À la création, la tâche est suspendue.
	\item \texttt{int detruire\_tache(int id)} : Détruit la tâche dont l'identifiant est passée en paramètre.
	
	\item \texttt{int demarrer(int id)} : Démarre la tâche dont l'identifiant est passé en paramètre.
	\item \texttt{int suspendre(int id)} : Suspend la tâche dont l'identifiant est passé en paramètre. 
	\item \texttt{int terminer(int id)} : Tue la tâche dont l'identifiant est passé en paramètre.
	
	\item \texttt{int changer\_priorite(int id, int priorite)} : Change la priorité de la tâche dont l'identifiant est passé en paramètre à la priorité passée en paramètre.

	\item \texttt{int obtenir\_id()} : Retourne l'identifiant de la tâche courante.
	\item \texttt{enum obtenir\_etat(int id)} : Retourne l'état de la tâche dont l'identifiant est passé en paramètre. Si l'identifiant passé en paramètre est invalide, \texttt{INEXISTANT} est renvoyé.
	\item \texttt{int obtenir\_priorite(int id)} : Retourne la priorité de la tâche dont l'identifiant est passé en paramètre.
\end{itemize}
}
