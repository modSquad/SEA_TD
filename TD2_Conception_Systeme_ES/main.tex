\documentclass[a4paper,11pt]{article}
\usepackage[english, french]{babel}
\usepackage[utf8]{inputenc}

\usepackage{graphicx}
\usepackage{fancyhdr}
\usepackage{lastpage}

\usepackage{hyperref}
\usepackage{float}

\usepackage[top=20mm, bottom=20mm, left=25mm, right=25mm]{geometry}
\pagestyle{fancy}

\renewcommand{\sfdefault}{phv}
\renewcommand{\rmdefault}{phv}
\renewcommand{\ttdefault}{pcr}

\newcommand{\nTitle}[1]
{
	\newpage			
	\vspace*{\fill}		
	\begin{center}	
		\part{#1}		
	\end{center}
	\vspace*{\fill}		
	\newpage			
}

\newenvironment{nAbstract}		
{								
	\vspace*{\fill}				
	\begin{center}				
		\begin{abstract}		
}
{
		\end{abstract}			
	\end{center}				
	\vspace*{\fill}				
}

%\usepackage{algo}
\usepackage{arydshln} % Array dashed lines
\usepackage{xspace} % Pour que les espaces soit bien gérés dans les autres packages,
		% à proximité des guillemets par exemple.
\usepackage{subfig}
\graphicspath{{./img/}}


%\usepackage{algo}
\usepackage{arydshln} % Array dashed lines
\usepackage{xspace} % Pour que les espaces soit bien gérés dans les autres packages,
		% à proximité des guillemets par exemple.
\usepackage{subfig}
\graphicspath{{./img/}}


\title{TD2 - Conception du système d'E/S}

\author{\textsc{Adenot} Paul, \textsc{Brodu Etienne}, \textsc{Gaudin} Maxime,\\ \textsc{Golumbeano} Monica, \textsc{Richard} Martin}
\lhead{TD2}
\cfoot{\thepage\ of \pageref{LastPage}}

\begin{document}
\maketitle
\tableofcontents 

\begin{nAbstract}
Réponse aux 5 questions du TD2 :
\begin{enumerate}
	\item Compléter le schéma
	\item Démontrer comment on peut "fabriquer" le mode \textsl{synchrone} sur le mode \textsl{asynchrone}
	\item \og Il y un bug\fg : Reprendre la solution en mode asynchrone et poser vous la question \og Comment le processus 
		appelant récupère l'@IORB pour le libérer ? Proposer une solution
	\item \textbf{p.14}, est-ce que le champ \og N tâche responsable de ls requête d'E/S\fg de la structure de données IORB 
	est nécessaire ? Justifier
   \item En algorithmie de haut niveau, donner l'algo de la commande shell \texttt{ls}	
\end{enumerate}
\end{nAbstract}

\newpage
\section{Q2}
Le mode synchrone peut être construit sur le mode asynchrone en plaçant
simplement un point de synchronisation sur la ressource (au sens large).
On placera alors juste un appel à \texttt{WAIT} sur le sémaphore \emph{requête
traité}, juste après l'appel de \texttt{MONES}. On aura là le comportement
d'un mode synchrone : l'exécution sera bloquée entre l'appel à la primitive
d'entrée sortie et le moment ou cette primitive d'entrée sortie retournera.


\section{Q3}
L'adresse du buffer de l'échange est contenu dans la structure \texttt{tab\_ech} et dans l'\texttt{IORB}, 
l'\texttt{IORB} n'est donc pas utile dans le contexte du processus utilisateur.
Le processus utilisateur ne doit pas pouvoir y accéder, d'autant plus que ce n'est pas lui qui l'instancie.
Cependant, la procédure \texttt{MONES\_ASYN} ne peut pas se charger de libérer l'IORB car il est utilisé hors du contexte de
la procédure.
La tache d'interruption en RMX devras donc libérer l'IORB lorsque elle n'en à plus besoin, après avoir envoyé
le signal sur le sémaphore requête servie.\\

Attention, il peut apparaître un conflit avec le numéro de la tâche responsable de l'échange puisque
il est potentiellement appelé après la destruction de l'\texttt{IORB} tel que décrite précedemment.
La question 4 répondra de l'utilité de ce numéro.

\section{Q4}
La tâche responsable de la requête d'E/S va créer un sémaphore de synchronisation, initialisé à 0 et le transmettre à
MONES\_ASYNC. C'est l'attente d'un jeton dans ce sémaphore qui va permettre de synchroniser la fin de la
requête d'E/S avec la tâche qui la demande. L'IORB contient l'adresse de ce sémaphore. La tâche d'intéruption
qui traite la requête ajoutera un jeton dans ce sémaphore une fois l'opération d'E/S réalisée.\\
˜\\
Le numéro de la tâche responsable de la requête n'est pas utilisé par le système de gestion d'entrées/sorties
du système, la seule communication effective entre le système d'E/S et la tâche est la synchronisation finale.
Ainsi, le sémaphore "requête servie" suffit pour remplir ce rôle.i

\section{Q5}
\begin{algo}
\ALGO{Procedure LS}
\CONST
	\DECLCONST{TAILLE\_POSTE}{Taille maximum d'un poste}
\ENDCONST
\VAR
	\DECLVAR{RepCourant}{Nom du répertoire courant}
	\DECLVAR{Fichiers}{Tableau de descripteurs de fichier du répertoire courant}
	\DECLVAR{N}{Nombre de postes du répertoire courrant}
\BEGIN
	\STATE{Initialiser RepCourant au répertoire courant}
	\STATE{Ouvrir le descripteur de fichier associé à RepCourant}

	\STATE{}
	\STATE{N \recoit Nombre de poste de RepCourant}
	\STATE{Initialiser Fichier ( Taille : N * TAILLE\_POSTE )}

	\STATE{}
	\FORGEN{F allant de 1 à N}
		\STATE{Fichiers[F] \recoit Descripteur du poste F}
		\STATE{Afficher sur la sortie standart les attributs de Fichier[F]}
	\ENDFOR

	\STATE{}
	\STATE{Libérer la mémoire}
\END
\end{algo}

\end{document}
