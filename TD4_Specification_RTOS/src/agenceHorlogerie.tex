\Agence{Agence horlogerie}
{
L'agence horlogerie fournie un service d'acquisition du temps.
Elle offre également la création d'évenement à une date ou après l'écoulement d'un temps.
}
{
\begin{itemize}
	\item La structure \lstinline {struct duree\_t} est composée de :
	\begin{itemize}
		\item seconde : \lstinline {int}
		\item mseconde : \lstinline {int} 
	\end{itemize}

	\item La structure \lstinline {struct time\_t} est composée de :
	\begin{itemize}
		\item heure : \lstinline {int}
		\item minute : \lstinline {int} 
		\item seconde : \lstinline {int}
		\item mseconde : \lstinline {int} 
	\end{itemize}

	\item La structure \lstinline {struct date\_t} est composée de :
	\begin{itemize}
		\item jour : \lstinline {int}
		\item mois : \lstinline {int}
		\item annee : \lstinline {int} 
	\end{itemize}

	\item La structure \lstinline {struct datetime\_t} est composée de :
	\begin{itemize}
		\item date : \lstinline {struct date\_t}
		\item time : \lstinline {struct time\_t} 
	\end{itemize}
\end{itemize}
}
{
\begin{itemize}

	\item \lstinline {time\_t getCurrentTime()}
	Cette primitive permet de récupérer le temps courant.

	\item \lstinline {date\_t getCurrentDate()}
	Cette primitive permet de récupérer la date courante.

	\item \lstinline {setCurrentTime(time\_t time)}
	Cette primitive permet de définir le temps courant.

	\item \lstinline {setCurrentDate(date\_t date)}
	Cette primitive permet de définir la date courante.

	\item \lstinline {event\_id createEvent (datetime\_t dateTime)}
	Créer un évenment qui sera déclenché à la date \texttt{dateTime}.

	\item \lstinline {event\_id createEvent (duree\_t duree)}
	Créer un évenement qui sera déclenché à la date courante + \texttt{duree}.
\end{itemize}
}
{
\subsubsection{Reflexion sur le rapport entre f1 et f2}

\begin{itemize}
\item \emph{f1} : fréquence des interruptions de l'horloge temps réel.
\item \emph{f2} : fréquence des temps minimum d'execution d'une tache avant préemption
\end{itemize}

La fréquence \emph{f1} est la fréquence d'execution des instruction cadencé par l'horloge temps réel.
Il n'est pas avantageux d'avoir une fréquence \emph{f2} trop proche de la fréquence d'execution des instructions, car le changement de contexte pourrais être alors plus consommateur en ressources que l'execution des tâches elle-même.

Plus \emph{f2} est proche de \emph{f1}, meilleur est le temps de réponse, car les tâches rendent la main très fréquement.
Il sera donc rare d'attendre une tâche longtemps.
En revanche, le temps d'execution sera rallongé du temps de changement de contexte fréquent.

Plus \emph{f2} est éloigné de \emph{f1}, meilleur est l'efficacité d'execution.
On économise tout les temps de passage d'une tâche à l'autre, en revanche, la réactivité est moins bonne.

Le temps moyen d'execution d'une primitive seras 10 à 100 fois plus long que le temps d'éxecution d'une instruction, car les primitives comportent plusieurs dizaine d'instructions.
Le temps moyen d'execution d'une tâche doit permettre à plusieurs primitive de s'executer, il sera donc environ 100 fois plus long que le temps d'execution d'une primitive.
}
