Agence horlogerie

\Object{Agence horlogerie}
{
L'agence horlogerie fournie un service d'acquisition et de gestion du temps.
Cette agence permet d'obtenir le temps écoulé depuis le 1 janvier 1979 à minuit à la milliseconde près.

Elle offre également la manipulation simple de chronomètres, la création d'évenement à une date ou après l'écoulement d'un temps.
}
{
struct duree
seconde
mseconde

struct time
heure
minute
seconde
mseconde

struct date
jour
mois
annee

struct datetime
date
time
}
{
getCurrentTime

getCurrentDate

Chronometre

	create chrono

	delete chrono

	getTime from chrono


Evenement

	create Event (datetime)

	create Event (duree)
}
{}
{
\subsubsection{Reflexion sur le rapport entre f1 et f2}

f1 fréquence des interruptions de l'horloge temps réel.
f2 fréquence des temps minimum d'execution d'une tache avant préemption

Plus f2 est proche de f1, meilleur est le temps de réponse.
On permet aux tâches de changer très rapidement entre elle, on couvre un maximum de tâches simultanément.
En revanche, il faudras plus de temps d'éxecution, étant donné que le passage entre 2 tâche consomme un certain temps processeur.

Plus f2 est éloigné de f1, meilleur est l'efficacité d'execution.
On économise tout les temps de passage d'une tâche à l'autre, en revanche, on perd en réactivité.

Le temps moyen d'execution d'une tâche.
Le temps maximum d'execution d'une primitive
Le temps moyen d'exécution d'une instruction



Il n'est pas avantageux d'avoir une fréquence f2 trop proche de la fréquence d'execution des instructions.

Le temps moyen d'execution d'une primitive seras 10 à 100 fois plus long que le temps d'éxecution d'une instruction.

Le temps moyen d'execution d'une tâche doit permettre à plusieurs primitive de s'executer, il sera donc environ 100 fois plus long que le temps d'execution d'une primitive.
}
