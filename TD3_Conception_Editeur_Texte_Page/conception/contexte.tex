La suite de ce document présente la démarche de conception entreprise dans le
cadre de ce projet. Nous souhaitons proposer un éditeur de textes page capable
de manipuler efficacement un fichier texte séquentiel de longueur variable.

L'architecture retenue à l'issue de cette phase de conception doit permettre
d'offrir un outil ergonomique (aspect essentiellement traité dans le cadre de
la spécification), portable sur différents environements et différentes architectures, maintenable et
efficace. Nous devons garder en mémoire que cet outil doit pouvoir être déployé
dans un environnement où les ressources sont limitées.

Nous analyserons et critiquerons trois solutions de gestion des données en
mémoire centrale et mémoire secondaire selon les critères de portabilité,
maintenabilité et efficacité. Nous retiendrons celle jugée la plus pertinente
et nous la décrirons en détail.
