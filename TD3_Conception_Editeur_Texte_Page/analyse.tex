\section{Analyse critique du mode opératoire}
Dans cette section nous allons comparer le mode opératoire appliqué pour la spécification de l'éditeur de page et les deux autres méthodes proposés pas J.M. \textsc{Pinon} et J.F. \textsc{Petit}.

\subsection{Comparaison avec la méthode de J.M. \textsc{Pinon}}
La méthode présentée par M. \textsc{Pinon} est basée sur une analyse très poussée des besoins et l'élaboration des différents diagrammes de cas d'utilisation et de séquences. Cette analyse permet d'avoir une vision exhaustive des besoins de l'utilisateur mais peut devenir assez lourde et relativement chronophage. 
	
En revanche le mode opératoire appliqué pour la réalisation de l'éditeur de texte part d'un étude moins systématique et permet d'établir beaucoup plus rapidement les principales fonctionnalités.

\subsection{Comparaison avec la méthode de J.F. \textsc{Petit}}
La méthode de J.F. \textsc{Petit} est basée sur les objets métiers. Pour extraire ces objets il est nécessaire de réaliser une étude de données. En principe on a un objet métier par fenêtre ce qui est assez rigide et peut dérouter l'utilisateur (le nombre de fenêtres peut devenir très grand).  Pourtant, cette méthode a l'avantage d'être assez facile à mettre en œuvre.
